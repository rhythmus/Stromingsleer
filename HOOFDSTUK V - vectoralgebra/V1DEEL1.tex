\documentclass[]{article}
\usepackage{lmodern}
\usepackage{amssymb,amsmath}
\usepackage{ifxetex,ifluatex}
\usepackage{fixltx2e} % provides \textsubscript
\ifnum 0\ifxetex 1\fi\ifluatex 1\fi=0 % if pdftex
  \usepackage[T1]{fontenc}
  \usepackage[utf8]{inputenc}
\else % if luatex or xelatex
  \ifxetex
    \usepackage{mathspec}
    \usepackage{xltxtra,xunicode}
  \else
    \usepackage{fontspec}
  \fi
  \defaultfontfeatures{Mapping=tex-text,Scale=MatchLowercase}
  \newcommand{\euro}{€}
\fi
% use upquote if available, for straight quotes in verbatim environments
\IfFileExists{upquote.sty}{\usepackage{upquote}}{}
% use microtype if available
\IfFileExists{microtype.sty}{%
\usepackage{microtype}
\UseMicrotypeSet[protrusion]{basicmath} % disable protrusion for tt fonts
}{}
\usepackage{longtable,booktabs}
\usepackage{graphicx}
\makeatletter
\def\maxwidth{\ifdim\Gin@nat@width>\linewidth\linewidth\else\Gin@nat@width\fi}
\def\maxheight{\ifdim\Gin@nat@height>\textheight\textheight\else\Gin@nat@height\fi}
\makeatother
% Scale images if necessary, so that they will not overflow the page
% margins by default, and it is still possible to overwrite the defaults
% using explicit options in \includegraphics[width, height, ...]{}
\setkeys{Gin}{width=\maxwidth,height=\maxheight,keepaspectratio}
\ifxetex
  \usepackage[setpagesize=false, % page size defined by xetex
              unicode=false, % unicode breaks when used with xetex
              xetex]{hyperref}
\else
  \usepackage[unicode=true]{hyperref}
\fi
\hypersetup{breaklinks=true,
            bookmarks=true,
            pdfauthor={},
            pdftitle={},
            colorlinks=true,
            citecolor=blue,
            urlcolor=blue,
            linkcolor=magenta,
            pdfborder={0 0 0}}
\urlstyle{same}  % don't use monospace font for urls
\setlength{\parindent}{0pt}
\setlength{\parskip}{6pt plus 2pt minus 1pt}
\setlength{\emergencystretch}{3em}  % prevent overfull lines
\setcounter{secnumdepth}{0}

\date{}

\begin{document}

\begin{enumerate}
\item
  \subsection{V.1.1.1. Scalaire
  grootheden.}\label{v.1.1.1.-scalaire-grootheden.}
\end{enumerate}

Grootheden die uitsluitend door een reëel getal gekenmerkt worden noemen
we scalaire grootheden. Ze worden symbolisch voorgesteld door een letter
en uitgedrukt in de gepaste S.I.-eenheden.

\emph{Een fysische scalaire grootheid zal dus aangeduid worden door:}

- een symbool;

- een getal (het maatgetal);

- een eenheid.

Voorbeelden uit de mechanica zijn:

een massa: m = 10 kg

een tijd: t = 10 s

\begin{enumerate}
\item
  \subsection{V.1.1.2. Vectoriële
  grootheden.}\label{v.1.1.2.-vectoriuxeble-grootheden.}
\end{enumerate}

In vectoriële grootheden vinden we bovenstaande kenmerken eveneens
terug. Ze volstaan echter niet om de grootheid volledig te kenmerken. Om
bijvoorbeeld een kracht volledig te bepalen moeten, naast haar grootte,
ook de richting en de zin waarin ze werkzaam is, aangegeven worden. We
spreken dan over een vectoriële grootheid.

\emph{Een fysische vectoriële grootheid wordt gekenmerkt door:}

- \emph{een grootte of norm}

Deze wordt aangegeven door een \emph{maatgetal} en een \emph{eenheid},
voorafgegaan door een symbool:

bijvoorbeeld: F = 100 N v = 10 m/s

- \emph{een richting}

Aan iedere vectoriële grootheid wordt een richting gekoppeld.

bijvoorbeeld.: - een kracht werkt verticaal.

- een auto volgt de richting van de E19

- \emph{een zin}

Om eenduidig te zijn moet aan de richting ook een zin worden gekoppeld.

bijvoorbeeld.: - een kracht werkt verticaal naar boven.

- een auto rijdt langs de E19 naar Brussel

Bij sommige vectoriële grootheden moet bovendien nog de ligging
aangegeven worden; zo heeft bijvoorbeeld het aangrijpingspunt van een
kracht belang wanneer men haar rotatie-effect wil bestuderen.

\emph{De grafische voorstelling van een vectoriële grootheid .}

Een vectoriële grootheid kan grafisch worden voorgesteld door een pijl,
die we \emph{de vector} noemen. (fig.1)

\emph{De grootte:} de afstand tussen het beginpunt a en het eindpunt b
of het maatgetal van de lengte van het lijnstuk ab. Met behulp van een
overeengekomen schaal kan men op die manier de grootte van een
vectoriële grootheid grafisch aangeven.

\emph{De richting:} de richting van de rechte ab die begin- en eindpunt
verbindt.

\emph{De zin:} wordt aangegeven door een pijltje te plaatsen van het
beginpunt (a) naar het eindpunt (b) van de vector.

De onbepaalde verlengde lijn waarvan de vector deel uitmaakt heet de
\emph{drager van de vector}. Stelt de vector een kracht voor dan noemt
men het beginpunt gewoonlijk het \emph{aangrijpingspunt}. De drager
noemt men dan gewoonlijk de \emph{werklijn} van de kracht.

Bij de grafische voorstelling worden de richting en de zin van de
vectoriële grootheid van de figuur afgelezen. Het symbool, de
getalwaarde en de eenheid worden bij de vector geschreven om aan te
duiden over welke fysische grootheid het precies gaat. De ligging van de
vector moet nog afzonderlijk worden aangegeven.

In figuur 2 worden op die manier een kracht voorgesteld van 10 N
verticaal naar boven en twee snelheden, allebei gelijk aan 4 m/s,
respectievelijk horizontaal naar rechts en horizontaal naar links. Merk
op dat bij geen minteken wordt geplaatst. Het pijltje geeft immers aan
dat de zin van de snelheid naar links gericht is.

\emph{Notaties.}

Om een duidelijk onderscheid te maken tussen scalaire en vectoriële
grootheden zullen we, voor vectoriële grootheden, de symbolen voorzien
van een \emph{pijltje}.

Voorbeelden:

figuur 1: een vector met beginpunt a en eindpunt b stellen we voor door
. We kunnen deze vector eventueel ook een andere naam geven,
bijvoorbeeld .

figuur 2: de snelheidsvectoren noteren we als en en de krachtvector als
.

Spreken we enkel over de grootte of de norm van de vector dan schrijven
we:

\begin{quote}
ab of u

F = 10 N

v1 = 4 m/s

v2 = 4 m/s
\end{quote}

Deze laatste notatie geeft niets over de richting en de zin van de
vector. We moeten dus nog naar een middel zoeken om deze richting en zin
op een eenvoudige maar éénduidige manier aan te geven. Dit kan slechts
na het invoeren van een zeer belangrijk begrip in de vectoralgebra,
namelijk de \emph{eenheidsvector} of de \emph{richtingsvector}.

We nemen een rechte en brengen hierop een normering aan. Dit betekent
dat we een zin bepalen (de zin van stijgende abscissen) en een
lengteëenheid (fig. 3). We hebben nu een genormeerde rechte of een as.
Gewoonlijk zullen we de normering weglaten en de positieve zin aangeven
bij het uiteinde van de as (fig. 4).

De eenheidsvector van deze as is een vector waarvan de richting en de
zin overeen komen met die van de as, terwijl zijn grootte gelijk is aan
de lengteëenheid. Men duidt hem aan door (fig.5), waarbij de index x de
aanduiding is van de as waarop de eenheidsvector gelegen is: hij geeft
de richting van alle assen die evenwijdig zijn aan de x-as. Voor elke
richting, zowel in een plat vlak als in de ruimte, kan zulke
eenheidsvector gedefinieerd worden.

In figuur 6 hebben de assen a en b dezelfde richting en dezelfde zin. As
c daarentegen is ook evenwijdig maar heeft de tegengestelde zin. We
noteren dit als volgt:

In figuur 7 zijn twee eenheidsvectoren voorgesteld respectievelijk
volgens een x- en een y-as die loodrecht op elkaar staan. Zoals later
zal blijken kunnen alle vectoren, gelegen in hetzelfde xy-vlak,
uitgedrukt worden in functie van deze twee eenheidsvectoren. We noemen
ze de \emph{hoofdeenheidsvectoren} of de \emph{hoofdrichtingsvectoren}.

In een driedimensionele ruimte zal er nog een derde hoofdeenheidsvector
aan toegevoegd worden die loodrecht staat op beide vorige (fig.8).

\emph{Opmerking:}

Een eenheidsvector is een dimensieloze grootheid. Hij kan dus gebruikt
worden om de richting aan te geven van om het even welke vectoriële
grootheid. Twee verschillende fysische vectoriële grootheden met
dezelfde richting hebben dus dezelfde richtingsvector.

Met behulp van het begrip eenheidsvector kan elke fysische vectoriële
grootheid weergegeven worden. Een vector , evenwijdig aan de x- as kan
steeds geschreven worden onder de gedaante:

a is hierin een reëel getal dat zowel positief als negatief kan zijn.

De getalwaarde van a geeft \emph{de grootte} van de vectoriële grootheid
aan. De vector geeft \emph{de richting} aan. Hij zegt dat de vector
evenwijdig is aan de x- as. Het teken van a geeft \emph{de zin} aan. Het
plusteken wil zeggen dat de vector dezelfde zin heeft als de x- as. Het
minteken wil zeggen dat de vector de tegengestelde zin heeft van de x-
as. Om aan te duiden over welke fysische grootheid het gaat, voegen we
er het symbool en de eenheid aan toe. De schrijfwijze van de vectoren
voorgesteld in figuur 10 is dus:

\begin{longtable}[c]{@{}lll@{}}
\toprule
& &\tabularnewline
& &\tabularnewline
\bottomrule
\end{longtable}

Het plusteken wordt gewoonlijk weggelaten.

\emph{Opmerkingen.}

1. Hoewel de krachten en op een verschillende plaats liggen worden ze op
dezelfde manier geschreven. De vectoriële schrijfwijze zegt dus niets
over de ligging van de vector. Heeft de ligging belang dan moet deze nog
bijkomend aangeduid worden.

2. Heeft een vector dezelfde richting als een willekeurige rechte pq
(fig. 11) dan kunnen we schrijven:

3. Deze vergelijking kan ook als volgt geschreven worden:

Wanneer een vector niet samenvalt met de richting van de x-as, de y-as
of de z-as dan wordt de eenheidsvector bekomen door de vector te delen
door zijn grootte.

De vector, zoals hij in de wiskunde gedefinieerd wordt, is een
\emph{vrije vector}. Dat betekent dat we het beginpunt gelijk waar mogen
kiezen in de ruimte. Zo zal in de mechanica blijken dat een koppelvector
gelijk waar mag geplaatst worden. Een koppelvector is dus een vrije
vector.

Een vector met een welbepaald aangrijpingspunt noemen we een \emph{vaste
of gebonden vector}, zoals bijvoorbeeld de vector die de plaats aanduidt
van een punt t.o.v. een referentiepunt.

Tenslotte kennen we ook \emph{glijdende vectoren}. Het zijn vectoren
waarvan we het beginpunt mogen verplaatsen op de drager, zoals
bijvoorbeeld een kracht.

Twee vectoren zijn gelijk als ze

- dezelfde richting

- dezelfde zin

- dezelfde grootte hebben. (fig. 12)

Schrijfwijze:

Twee vaste vectoren zijn gelijk als ze

- dezelfde richting

- dezelfde zin

- dezelfde grootte

- hetzelfde beginpunt hebben. (fig. 13)

\begin{quote}
Schrijfwijze:
\end{quote}

Twee vectoren zijn tegengesteld als ze

- dezelfde richting

- dezelfde grootte

- de tegengestelde zin hebben. (fig. 14)

\begin{quote}
Schrijfwijze:
\end{quote}

Twee vectoren zijn rechtstreeks tegengesteld als het vaste of glijdende
vectoren zijn met

- dezelfde richting

- dezelfde grootte

- de tegengestelde zin

- en gelegen zijn op dezelfde drager. (fig. 15)

\begin{quote}
Schrijfwijze:
\end{quote}

Indien het gaat om fysische vectoriële grootheden moet het uiteraard
gaan om vectoren van dezelfde aard uitgedrukt in dezelfde eenheden.

Het product van een gegeven vector met een scalaire waarde m (een
algebraïsch getal) is een nieuwe vector m met dezelfde richting maar met
een lengte die gelijk is aan m maal de lengte van de gegeven vector.

De zin is dezelfde als die van indien m positief is en tegengesteld aan
die van indien m negatief is.

Is een vaste vector dan moet de vector m ook hetzelfde beginpunt hebben.

De deling van een vector door een algebraïsch getal komt overeen met een
vermenigvuldiging met het omgekeerde van dat getal.

Figuur 16 stelt voor: a. de vector

b. de vector 2

c. de vector -3

De hoek tussen twee vectoren is de kleinste hoek tussen de twee
positieve richtingen. In figuur 17 stelt θ de hoek voor tussen de
vectoren en .

Indien beide vectoren niet in hetzelfde punt aangrijpen verplaatst men
(eventueel in zijn verbeelding) één van de twee vectoren naar het
aangrijpingspunt van de andere.

In figuur 18 is de hoek θ tussen de twee vectoren en kleiner dan 90°.

In figuur 19 is de hoek θ tussen de vectoren en groter dan 90°.

1. De som van twee vrije vectoren en is een vrije vector die geschreven
wordt als

, en die als volgt wordt bepaald: (fig. 20)

- in een willekeurig punt o plaatsen we de vector

- in zijn eindpunt plaatsen we het beginpunt van de vector

- de vector met beginpunt in het beginpunt van de eerste vector en
eindpunt in het eindpunt van de tweede is de vectorsom van de twee
vectoren.

\begin{quote}
Deze vectorsom is ook te beschouwen als de diagonaal van het
parallellogram geconstrueerd met en als zijden (fig. 21).
\end{quote}

2. De vectoriële som van k willekeurige vectoren , , , ... , is een
vrije vector waarvan men schrijft:

\begin{quote}
Hij wordt op een gelijkaardige manier bepaald: (fig. 22)
\end{quote}

- in een willekeurig punt o plaatsen we de vector

- in zijn eindpunt plaatsen we het beginpunt van de vector

- in het eindpunt van , het beginpunt van , enz.

- de vector , met beginpunt in het beginpunt van de eerste vector en
eindpunt in het eindpunt van de laatste is de vectorsom van de
beschouwde vectoren.

\emph{Eigenschappen van de vectoriële som.}

1. De som van vectoren is commutatief, d.w.z dat de som onafhankelijk is
van de volgorde waarin de vectoren gesommeerd worden:

2. De vectoriële som is associatief, d.w.z.dat de som ongewijzigd blijft
indien men een aantal vectoren door hun som vervangt:

3. Het product van een vector met een algebraïsche som (m+n+p) is gelijk
aan de som der vectoren m, n, p:

Het vectorieel verschil wordt beschouwd als de som der vectoren en (fig.
23).

De eigenschappen van het vectorieel verschil zijn gelijkaardig aan deze
van de vectoriële som.

De grootte van de vectoriële som (fig. 24):

De grootte van het vectorieel verschil (fig. 25):

In de uitdrukking van de grootte van de som staat dus altijd een
plusteken en voor een verschil altijd een minteken op voorwaarde dat de
hoek \includegraphics{media/image26.wmf} de hoek is tussen de vectoren
en zoals deze in paragraaf V.1.7. werd gedefinieerd.

\emph{Voorbeelden:}

1. Bepaal de afstand bc in figuur 26.

\emph{Oplossing:}

We maken er op één of andere manier een vectoriële betrekking van (fig.
27).

2. Bepaal de afstand ac in de driehoek van figuur 28.

\emph{Oplossing:}

We maken er terug een vectoriële betrekking van.

\begin{enumerate}
\def\labelenumi{\alph{enumi}.}
\item
  In figuur 29 stellen we voor:
\end{enumerate}

b. In figuur 30 stellen we voor:

\end{document}
